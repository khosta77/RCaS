\documentclass[a4paper, 17pt]{report}


\usepackage{cmap}
\usepackage[T2A]{fontenc}
\usepackage[utf8]{inputenc} 
\usepackage [english, russian]{babel} 

% \documentclass{article}
\usepackage{graphicx}
\graphicspath{ {images/ } }

\usepackage{amsmath}
\usepackage{mathtools}

\usepackage{tikz,filecontents,amsmath}
\usepackage{pgfplots}
\pgfplotsset{width=11cm,compat=1.18}


\begin{document}
\textbf{ \textsc{Параграф 5 Задача 2}}

\textbf{Дано:}

$\theta = 20 \cdot 10^{-6} $ [c]

$U_0 = 15 $ [В]

\textbf{Решение:}

\centerline{
 $ S(t)=\frac{S_0 \cdot \omega_0}{\pi} \cdot \frac{\text{sin} \omega_0 \cdot t }{\omega_0 \cdot t} $
}
, где \Large $\frac{\text{sin} \omega_0 \cdot t }{\omega_0 \cdot t} = 1$ , из этого следует, что $\text{t} = \pm \frac{\pi}{\omega_0}$ , тогда

\centerline{  \Large $\text{t}_1 = \pm \frac{\pi}{\omega_0}$ и $\text{t}_2 = \frac{-\pi}{\omega_0}$.} 
И

\centerline{ \Large  $U_0 = \frac{S_0 \cdot \omega_0}{\pi}$.}

1) Из графика видно, что:

\centerline{
	\Large $\frac{\pi}{\omega_0}=\frac{\theta}{2} \Rightarrow \omega_0 = \frac{2\cdot\pi}{\theta} = \frac{2\cdot\pi}{20 \cdot 10^{-6}} = 0,314 \cdot 10^6 \text{ [с}^{-1}\text{]}$
}
, тогда ширина полосы частот:

\centerline{
	\Large $\triangle\omega = \omega_0 \cdot 2 = 0,314 \cdot 10^6 \cdot 2 = 0,628 \cdot 10^6 \text{ [с}^{-1}\text{]}$
}

2) Вычислим $S_0$:

\centerline{
	\Large $U_0 = \frac{S_0 \cdot \omega_0}{\pi} \Rightarrow$
}
	
\centerline{
	\Large $S_0 = \frac{U_0 \cdot \pi}{\omega_0} = \frac{15 \cdot \pi}{0,314 \cdot 10^6} = 15 \cdot 10^{-5} \text{[Дж].}$
}





\end{document}
